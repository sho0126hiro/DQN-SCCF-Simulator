\documentclass[11pt,a4paper]{jsarticle}
%
\usepackage{amsmath,amssymb}
\usepackage{bm}
\usepackage{graphicx}
\usepackage{ascmac}
%
\setlength{\textwidth}{\fullwidth}
\setlength{\textheight}{39\baselineskip}
\addtolength{\textheight}{\topskip}
\setlength{\voffset}{-0.5in}
\setlength{\headsep}{0.3in}
%
\newcommand{\divergence}{\mathrm{div}\,}  %ダイバージェンス
\newcommand{\grad}{\mathrm{grad}\,}  %グラディエント
\newcommand{\rot}{\mathrm{rot}\,}  %ローテーション
%
\pagestyle{myheadings}
\markright{\footnotesize \sf 2020/07 1AS 廣瀬翔 \ 実験設定}
\begin{document}
%
%
\section*{シミュレータを用いたSCCFの性能調査 実験設定}
\section{実験概要}
2019年度実施したSCCFの特性調査実験では、カテゴリ数が状態行動数の削減のために6個であることや、2回前までの発話内容を格納するなど、状態行動数が少ない状態での学習ができることを確認した。

本実験では、状態行動数を増やし、深層強化学習アルゴリズムをSCCFに適用することで学習が可能であるかについて、性能を調査する。

ただし、実験を簡単にすることで、サイクルを早くするため、IDAは実際の利用者ではなく、エージェントを評価するシミュレータを対象として実験を実施する。
\section{エージェントのパラメータ設定}
\section{実験設定}
\section{シミュレータの設定}

%
%
\end{document}